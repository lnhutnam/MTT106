\section{Bài 06}

Xemx xét bài toán 
\begin{equation}
    \begin{aligned}
        \min \quad & \ln x - y\\
        \textrm{s.t.} \quad & x^2 + y^2 \leq 4\\
          & x \geq 1\\
    \end{aligned}
\end{equation}

\begin{enumerate}[label=(\alph*)]
    \item Tìm tất cả các điểm thỏa mãn điều kiện FJ.
    \item Tìm tất cả các điểm thỏa mãn điều kiện KKT.
    \item Những điểm KKT nào mà có giá trị mục tiêu thấp nhất (lowest objective value)?
    \item Xác định liệu một thỏa mãn điều kiện đủ cấp hai được thỏa mãn tại mỗi điểm KKT.
\end{enumerate}

\begin{solution}

    Các thành phần trong bài toán này như sau:
    \begin{align}
        \begin{aligned}
            f(x,y) &= \ln x - y\\
            g_1(x,y) &= x^2 + y^2 - 4\\
            g_2(x, y) &= -x + 1\\
        \end{aligned}
    \end{align}

    \begin{enumerate}[label=(\alph*)]
        \item Tìm tất cả các điểm thỏa mãn điều kiện FJ. 

        Ta hình thành dạng yếu của hàm Lagrangian như sau:
        \begin{equation}
            L(x, y, \lambda) = \lambda_0(\ln x - y) + \lambda_1(x^2 + y^2 - 4) - \lambda_2(x + 1)
        \end{equation}
        và các điều kiện FJ
        \begin{itemize}
            \item \begin{equation}
                \dfrac{\lambda_0}{x} + 2x\lambda_1 - \lambda_2 = 0
            \end{equation}
            \item \begin{equation}
                -\lambda_0 + 2y\lambda_1 = 0
            \end{equation}
            \item \begin{equation}
                \lambda_1 \geq 0, x^2 + y^2 - 4 \leq 0, \lambda_1(x^2 + y^2 - 4) = 0
            \end{equation}
            \item \begin{equation}
                \lambda_2 \geq 0, -x + 1 \leq 0, -\lambda_2(x-1) = 0
            \end{equation}
        \end{itemize}
        Giả sử $\lambda_0 = 0$, ta có:
        \begin{equation}
            \begin{cases}
                2x\lambda_1 - \lambda_2 = 0 \Leftrightarrow x = \dfrac{\lambda_2}{2\lambda_1}\\
                2y\lambda_1 = 0\\
            \end{cases}
        \end{equation}
        Điều này cho thấy $\lambda_1$ không thể bằng 0. Do đó, $y$ phải bằng 0. Thay $x$ vào biểu thức (6.7), ta tính được $\lambda_2 = 0$ hoặc $\lambda_2 = 2\lambda_1$. Trong cả trường hợp hai, ta được $x = 1$. Vậy điểm $(x, y) = (1, 0)$ là một điểm thỏa mãn điều kiện FJ. 
        
        Giả sử $\lambda_0 \ne 0$. Ta xét từng trường hợp dấu của các nhân tử
        \begin{itemize}
            \item TH1 ($\lambda_1 > 0, \lambda_2 > 0$): Từ biểu thức (6.6) và (6.7) ta có hệ 
            \begin{equation}
                \begin{cases}
                    x^2 + y^2 - 4 = 0\\
                    x - 1 = 0\\
                \end{cases}
                \Leftrightarrow 
                \begin{cases}
                    x = 1\\
                    y = \pm \sqrt{3}
                \end{cases}
            \end{equation}
            Các điểm $(x, y) = (1, \sqrt{3})$ và $(x, y) = (1, -\sqrt{3})$ thỏa mãn điều kiện FJ.
            \item TH2 ($\lambda_1 = 0, \lambda_2 = 0$): Từ biểu thức (6.4) và (6.5) ta có:
            \begin{equation}
                \begin{cases}
                    \dfrac{\lambda_0}{x} = 0\\
                    -\lambda_0 = 0
                \end{cases}
            \end{equation}
            Trường hợp này vi phạm điều kiện $\lambda_0 \ne 0$.
            \item TH3 ($\lambda_1 > 0, \lambda_2 = 0$). Từ biểu thức (6.4), (6.5), và (6.6), ta có
            \begin{equation}
                \begin{cases}
                    x^2 + y^2 = 4\\ 
                    y = \dfrac{\lambda_0}{2\lambda_1}\\
                    x = -\dfrac{\lambda_0}{2\lambda_1}\\
                \end{cases}
            \end{equation}
            Giải hệ này, ta được nghiệm $(x, y) = (\sqrt{2}, -\sqrt{2})$ thỏa mãn.
            \item TH4 ($\lambda_1 = 0, \lambda_2 > 0$). Từ biểu thức (6.5), $\lambda_1$ không thể bằng 0. Do đó, ta không tìm được điểm nào thỏa mãn FJ trong trường hợp này.
        \end{itemize}
        Vậy, các điểm thỏa mãn điều kiện FJ là $(x, y) = (\sqrt{2}, -\sqrt{2})$, $(x, y) = (1, 0)$, $(x, y) = (1, \sqrt{3})$ và $(x, y) = (1, -\sqrt{3})$.
        \item Tìm tất cả các điểm thỏa mãn điều kiện KKT.
        Ta hình thành hàm Lagrangian như sau:
        \begin{equation}
            L(x, y, \lambda) = (\ln x - y) + \lambda_1(x^2 + y^2 - 4) - \lambda_2(x + 1)
        \end{equation}
        và các điều kiện FJ
        \begin{itemize}
            \item \begin{equation}
                \dfrac{\partial L}{\partial x} = \dfrac{1}{x} + 2\lambda_1x - \lambda_2 = 0 
            \end{equation}
            \item \begin{equation}
                \dfrac{\partial L}{\partial y} = -1 + 2\lambda_1y = 0 \Leftrightarrow y = \dfrac{1}{2\lambda_1}
            \end{equation}
            \item \begin{equation}
                \lambda_1 \geq 0, x^2 + y^2 - 4 \leq 0, \lambda_1(x^2 + y^2 - 4) = 0
            \end{equation}
            \item \begin{equation}
                \lambda_2 \geq 0, -x + 1 \leq 0, -\lambda_2(x-1) = 0
            \end{equation}
        \end{itemize}
        Ta xét từng trường hợp dấu của các nhân tử
        \begin{itemize}
            \item TH1 ($\lambda_1 > 0, \lambda_2 > 0$). Từ biểu thức (6.15), và (6.16) ta có hệ
            \begin{equation}
                \begin{cases}
                    x^2 + y^2 = 4\\
                    x = 1
                \end{cases}
                \Leftrightarrow 
                \begin{cases}
                    x = 1\\
                    y = \pm\sqrt{3}
                \end{cases}
            \end{equation}
            Các điểm $(x, y) = (1, \sqrt{3})$ và $(x, y) = (1, -\sqrt{3})$ thỏa mãn điều kiện KKT.
            \item TH2 ($\lambda_1 = 0, \lambda_2 = 0$)
            Từ biểu thức (6.13) và (6.14) ta có:
            \begin{equation}
                \begin{cases}
                    \dfrac{1}{x} = 0\\
                    -1 = 0
                \end{cases}
            \end{equation}
            Hệ này vô lý nên không tìm được điểm nào thỏa mãn điều kiện KKT.
            \item TH3 ($\lambda_1 > 0, \lambda_2 = 0$). Từ biểu thức (6.13), (6.14), và (6.15), ta có
            \begin{equation}
                \begin{cases}
                    x^2 + y^2 = 4\\ 
                    y = \dfrac{1}{2\lambda_1}\\
                    x = -\dfrac{1}{2\lambda_1}\\
                \end{cases}
            \end{equation}
            Giải này ta thu được $\lambda_1 = \dfrac{\sqrt{2}}{4}$. Ta có điểm $(x, y) = (\sqrt{2}, -\sqrt{2})$ thỏa mãn điều kiện KKT.
            \item TH4 ($\lambda_1 = 0, \lambda_2 > 0$). Từ biểu thức (6.14), ta không thể tính được khi $\lambda_1 = 0$, nên ta không tìm được điểm nào thỏa mãn điều kiện KKT trong trường hợp này.
        \end{itemize}
        Vậy, các điểm $(x, y) = (\sqrt{2}, -\sqrt{2})$, $(x, y) = (1, \sqrt{3})$ và $(x, y) = (1, -\sqrt{3})$ là điểm thỏa mãn điều kiện KKT.
        \item Những điểm KKT nào mà có giá trị mục tiêu thấp nhất (lowest objective value)? Ta có giá trị hàm mục tiêu tại các điểm KKT như sau:
        \begin{itemize}
            \item $f(\sqrt{2}, -\sqrt{2}) = 1.7608$
            \item $f(1, -\sqrt{3}) = -\sqrt{3}$
            \item $f(1, -\sqrt{3}) = \sqrt{3}$
        \end{itemize}
        Vậy tại điểm $(x,y) = (1, -\sqrt{3})$ giá trị mục tiêu thấp nhất (lowest objective value).
        \item Xác định liệu một thỏa mãn điều kiện đủ cấp hai được thỏa mãn tại mỗi điểm KKT. \textbf{Đây là câu về điều kiện cấp hai.}
    \end{enumerate}
\end{solution}
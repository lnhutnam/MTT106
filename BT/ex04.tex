\section{Bài 04}

Trong bài toán,
\begin{equation}
    \begin{aligned}
        \min \quad & \sum_{j=1}^n\frac{c_j}{x_j}\\
        \textrm{s.t.} \quad & \sum_{j=1}^na_jx_j = b,\\
          &x_j \geq 0, j = 1, \dots, n,  \\
    \end{aligned}
\end{equation}
trong đó $a_j, c_j, b$ là các hằng số dương. Viết điều kiện FJ, và nếu áp dụng điều kiện KKT. Sau đó giải (các) nghiệm tối ưu $x^{*} = (x^{*}_1, x^{*}_2, \dots, x^{*}_n)$.

\begin{solution}

    Các thành phần của bài toán
    \begin{equation}
        \begin{cases}
            f(x) = \sum_{j=1}^n\frac{c_j}{x_j} \\
            h_1(x) = \sum_{j=1}^na_jx_j - b
        \end{cases}
    \end{equation}
    Ta thử viết các điều kiện Fritz John. Bằng cách tính toán, ta hình thành dạng yếu của hàm Lagrangian như sau:
    \begin{equation}
        L = \lambda_0\left(\sum_{j=1}^n\frac{c_j}{x_j}\right) + \mu_1\left(\sum_{j=1}^na_jx_j - b\right)
    \end{equation}
    Và ta viết được các điều kiện FJ như sau:
    \begin{enumerate}[label=(\alph*)]
        \item \begin{equation}
            -\lambda_0\left(\sum_{j=1}^n\frac{c_j}{x^2_j}\right) + \mu_1\left(\sum_{j=1}^na_j\right) = 0
        \end{equation}
    \end{enumerate}
    Nếu $\lambda_0 = 0$, thì $\mu_1 = 0$, điều này dẫn đến bài toán vô nghĩa. Thế nên, $\lambda_0 \ne 0$. Ta sử dụng hàm Lagrangian và viết các điều kiện KKT để dễ dàng hơn trong việc giải quyết bài toán này. Ta có hàm Lagrangian như sau:
    \begin{equation}
        L(x, \lambda) = \sum_{j=1}^n\frac{c_j}{x_j} + \mu_1\left(\sum_{j=1}^na_jx_j - b\right)
    \end{equation}
    và các điều kiện KKT được viết như sau:
    \begin{enumerate}[label=(\alph*)]
        \item \begin{equation}
            \dfrac{\partial L}{\partial x_j} = -\left(\sum_{j=1}^n\frac{c_j}{x^2_j}\right) + \mu_1\left(\sum_{j=1}^na_j\right) = 0, 
        \end{equation}
        \item \begin{equation}
            \sum_{j=1}^na_jx_j - b = 0, 
        \end{equation}
    Tính tổng biểu thức (4.6), ta có:
    \begin{equation}
        -\left(\sum_{j=1}^nc_j\right)\left(\sum_{j=1}^nx^2_j\right)^{-1} + \mu_1\left(\sum_{j=1}^na_j\right) = -\mathbf{C}\mathbf{x}^{-1} + \mu_1\mathbf{A} = 0\Leftrightarrow \mathbf{x} = \dfrac{\mathbf{C}}{\mu_1\mathbf{A}}
    \end{equation}
    Và dựa trên biểu thức (4.7), ta có:
    \begin{equation}
        \sum_{j=1}^na_jx_j - b = \left(\sum_{j=1}^na_j\right)\left(\sum_{j=1}^nx_j\right) - b = \mathbf{A}\mathbf{x} - b = \mathbf{A}\dfrac{\mu_1\mathbf{A}}{\mathbf{C}} - b = 0
    \end{equation}
    Ta tính được:
    \begin{equation}
        \mu_1 = \dfrac{\mathbf{C}b}{\mathbf{A}^2}
    \end{equation}
    Suy ra được:
    \begin{equation}
        \mathbf{x} = \dfrac{\mathbf{A}}{b} \quad\text{hay}\quad b\mathbf{x} = \mathbf{A}
    \end{equation}
    Ta giải phương trình $b\mathbf{x} = \mathbf{A}$ thì thu được nghiệm tối ưu $x^{*} = (x^{*}_1, x^{*}_2, \dots, x^{*}_n)$.
    \end{enumerate}
\end{solution}